\chapter{Atmospheric Large Eddy Simulation}

\section{Turbulent flows}
Under the right conditions, a fluid (or gas) in motion exhibits circular patterns that vary in size. A flow that features such patterns is described as a \emph{turbulent flow}, and in turbulence theory, the circular patterns are called \emph{eddies} \citep{popeTurbulentFlows2000}. We, as mankind, are surrounded by these kinds of flows in our daily lives. Examples include the flow of water through pipes, airflow around a moving car, waterfalls and Earth's atmosphere. Because turbulent flows are so predominantly present in our living environment, they are studied intensively by the scientific community and the industry. With these insights, accurate computer models are developed that attempt to predict the effects of turbulence.

To characterize turbulent flows, several scaling laws and non-dimensional numbers have been derived. The \emph{Reynolds number} $\text{Re}$ is defined as the ratio between inertial forces and viscous forces in a fluid: 

\begin{equation}
    \text{Re} = \frac{\text{Inertial forces}}{\text{Viscous forces}} = \frac{u \mathcal{L}}{\nu}
\end{equation}

in which $u$ is the flow speed, $\mathcal{L}$ is a characteristic length scale of the flow, usually determined by the geometry of the flow, and $\nu$ is the kinematic viscosity of the flow. In turbulent flows, the inertial forces dominate and the Reynolds number is therefore high ($>4000$ \citep{popeTurbulentFlows2000}). 

In turbulent flows, the larger an eddy, the more unstable it is. Therefore, large eddies tend to break up into slightly smaller eddies. These smaller eddies are once again unstable and beak up into even smaller eddies. This phenomenon is repeated until the remaining eddies are small enough such that the viscous forces in the fluid are able to convert the remaining kinetic energy into heat. The length scale at which the conversion of kinetic energy into heat happens is called the \emph{Kolmogorov length scale}, and is given by:

\begin{equation}
    \eta = \left( \frac{\nu^3}{\varepsilon} \right)^{1/4}
    \label{eq:kolmogorov-length}
\end{equation}

where $\varepsilon$ is the rate of dissipation of kinetic energy.


The objective of solving \autoref{eq:navier_stokes} such that eddies of all scales are represented imposes two requirements on the computational mesh. First, the mesh has to span a large enough area such that the largest scales can be captured. Second, the spacing between elements on the mesh must be at least as small as the Kolmogorov length scale of the flow. If both requirements are met, all turbulent motions can be resolved and no parameterizations are needed. This technique is called \acrfull{dns} \citep{popeTurbulentFlows2000}. The biggest limitation of DNS is that it is associated with extreme computational costs. ...

\citet{smagorinskyGeneralCirculationExperiments1963} proposed the idea of resolving only the largest, most energetic eddies and modeling the dissipation of energy at the smaller scales. This technique is now known as \acrfull{les}. To arrive at the governing equations of \acrshort{les}, the velocity field $\mathbf{u}(\mathbf{x},t)$ is first decomposed into a resolved part $\overline{\mathbf{u}}(\mathbf{x},t)$ and an unresolved part $\mathbf{u}^\prime(\mathbf{x},t)$. The unresolved part of the velocity field is often referred to as the \emph{subgrid-scale} component, as it represents the turbulent motions that are too small to be represented on the computational grid. This decomposition is analogous to a filtering operation; the small-scale highly fluctuating motions are filtered out of the flow field. The filtered Navier-Stokes equations are:

\begin{align}
    &\frac{\partial \overline{u}_i}{\partial x_j} = 0 \label{eq:continuity_filtered}\\ 
    &\frac{\partial \overline{u}_i}{\partial t} + \frac{\partial \overline{u_i u_j}}{\partial x_j} = - \frac{1}{\rho} \frac{\partial \overline{p}}{\partial x_i} + \nu \frac{\partial^2 \overline{u}_i}{\partial x_j} + f_i - \frac{\partial \tau_{ij}}{\partial x_j}.  \label{eq:momentum_filtered}
\end{align}

After the filtering operation, the momentum equation (\autoref{eq:momentum_filtered}) contains an extra term $\partial \tau_{ij} / \partial x_j$. This term is called the \emph{subgrid-scale stress tensor} and represents, as the name implies, the contribution of the unresolved scales to the resolved flow field. 


NS equations have to be solved numerically

Turbulence 

Solving Turbulence

DNS vs RANS vs LES

\section{The future of numerical weather prediction}

\section{DALES}
The \acrfull{dales} model is a large-eddy simulation model designed for high-resolution modeling of the atmosphere and its processes, like cloud formation and precipitation \citep{heusFormulationDutchAtmospheric2010,ouwerslootLargeEddySimulationComparison2017}. 

\subsection{Prognostic equations}
\acrshort{dales} uses six prognostic variables to define the state of the atmosphere: three velocity components $u$, $v$ and $w$, liquid water potential temperature $\theta_l$, total water specific humidity $q_t$ and the turbulence kinetic energy $e$. 

\begin{align}
    \frac{\partial \overline{u}_i}{\partial t} &= - \frac{\partial \overline{u}_i \overline{u}_j}{\partial x_j} - \frac{\partial \pi}{\partial x_i} + \frac{g}{\theta_0}\overline{\theta}_v \delta_{i3} + f_i - \frac{\partial \tau_{ij}}{\partial x_j} \label{eq:momentum_conservation}
\end{align}

where $u$ is the velocity vector, $\pi$ is a modified pressure \todo{find out what this means}, $\theta$ is the virtual potential temperature, .... Similarly, the momentum equation for a scalar $\varphi$, where $\varphi \in \{\theta_l, q_t\}$, is given by:

\begin{equation}
    \frac{\partial \overline{\varphi}}{\partial t} = - \frac{\partial \overline{u}_j \overline{\varphi}}{\partial x_j} - \frac{\partial R_{u_j,\varphi}}{\partial x_j} + S_\varphi,
\end{equation}

in which $R_{u_j,\varphi}$ is a sub-filter scale flux and $S_\varphi$ is a source term.

The budget equation for turbulence kinetic energy $e$ is given by:

\begin{equation}
\begin{split}
    \frac{\partial e^{1/2}}{\partial t} = &- \overline{u}_j \frac{\partial e^{1/2}}{\partial x_j} + \frac{1}{2e^{1/2}} \left( K_m \left( \frac{\partial \overline{u}_j}{\partial x_i} + \frac{\partial \overline{u}_i}{\partial x_j} \right) \frac{\partial \overline{u}_i}{\partial x_j} - K_h \frac{g}{\theta_0} \frac{\partial}{\partial z} \left( A \overline{\theta}_l + B \overline{q}_t\right) \right) \\ &+ \frac{\partial}{\partial x_j} \left( 2 K_m \frac{\partial e^{1/2}}{\partial x_j}\right) - \frac{c_{\varepsilon} e}{2 \lambda}
\end{split}
\end{equation}


\subsection{Sub-grid scale model}


\subsection{Mass conservation} \label{sec:dales_poisson}
In \acrshort{dales}, the atmosphere is modelled as an incompressible flow, which results in the following expression for conservation of mass:

\begin{equation}
    \frac{\partial \overline{u}_i}{\partial x_i} = 0. \label{eq:mass_conservation}
\end{equation}

Note that \autoref{eq:mass_conservation} states that the divergence of the velocity field $u_i$ must be equal to zero. This condition can be enforced by using Chorin's projection method \citep{chorinNumericalSolutionNavierStokes1967}. Per Chorin's method, the time integration of the left-hand side of \autoref{eq:momentum_conservation} is split into two parts. First, an intermediate velocity field $\overline{u}^*_i |_{t=t_n + \Delta t}$ is calculated that is not divergence-free (i.e., does not enforce mass balance) by evaluating all right-hand terms of \autoref{eq:momentum_conservation}: 

\begin{equation}
     = - \frac{\partial \overline{u}_i \overline{u}_j}{\partial x_j} + \frac{g}{\theta_0}\overline{\theta}_v \delta_{i3} + f_i - \frac{\partial \tau_{ij}}{\partial x_j} \label{eq:chorin_step1}
\end{equation}

In order to obtain an explicit equation for the pressure, the divergence of \autoref{eq:momentum_conservation} can be taken, which results in:

\begin{equation}
    \frac{\partial}{\partial x_i}\frac{\partial \overline{u}_i}{\partial t} = \frac{\partial}{\partial x_i} \left( - \frac{\partial \overline{u}_i \overline{u}_j}{\partial x_j} - \frac{\partial \pi}{\partial x_i} + \frac{g}{\theta_0}\overline{\theta}_v \delta_{i3} + \mathcal{F}_i - \frac{\partial \tau_{ij}}{\partial x_j} \right) \label{eq:momentum_divergence}
\end{equation}

\noindent Because of the symmetry of second derivatives, the order of the derivatives on the left-hand side can be inverted:

\begin{equation*}
    \frac{\partial}{\partial x_i}\frac{\partial \overline{u}_i}{\partial t} = \frac{\partial}{\partial t}\frac{\partial \overline{u}_i}{\partial x_i}
\end{equation*}

\noindent After applying the equation for conservation of mass (\autoref{eq:mass_conservation}):

\begin{equation}
    \frac{\partial}{\partial t} \frac{\partial \overline{u}_i}{\partial x_i} = 0
\end{equation}

\noindent Hence, \autoref{eq:momentum_divergence} can be rewritten as follows:

\begin{equation}
    \frac{\partial^2 \pi}{\partial x_i^2} = \frac{\partial}{\partial x_i} \left( - \frac{\partial \overline{u}_i \overline{u}_j}{\partial x_j} + \frac{g}{\theta_0}\overline{\theta}_v \delta_{i3} + \mathcal{F}_i - \frac{\partial \tau_{ij}}{\partial x_j} \right)
    \label{eq:poisson_equation}
\end{equation}

\noindent Since the right-hand side of \autoref{eq:poisson_equation} is known, it is essentially an equation of the form $\nabla^2 \Phi = f$, also known as the Poisson equation. This equation can efficiently be solved by making use of Fourier transfers. The first step towards a numerical solution to the Poisson equation is to discretize the equation. DALES uses a second-order finite difference scheme:

\begin{equation}
    \frac{\Phi_{i-1,j,k} - 2 \Phi_{i,j,k} + \Phi_{i+1,j,k}}{\Delta x^2} + \frac{\Phi_{i,j-1,k} - 2 \Phi_{i,j,k} + \Phi_{i,j+1,k}}{\Delta y^2} + \frac{\Phi_{i,j,k-1} - 2 \Phi_{i,j,k} + \Phi_{i,j,k+1}}{\Delta z^2} = f_{i,j,k}
    \label{eq:discretized_poisson_equation}
\end{equation}

\autoref{eq:discretized_poisson_equation} can be written as a matrix equation of the form $A \Phi = \mathbf{f}$, where $A$ has 7 non-zero components per row. Next, a Fourier transform is done in both horizontal directions, leading to the following equation: \todo{Add proof in appendix?}

\begin{equation}
    \left( \frac{\lambda_i}{\Delta x^2} + \frac{\lambda_j}{\Delta y^2} \right) \hat{\hat{\Phi}}_{i,j,k} + \frac{\hat{\hat{\Phi}}_{i,j,k-1} - 2 \hat{\hat{\Phi}}_{i,j,k} + \hat{\hat{\Phi}}_{i,j,k+1}}{\Delta z^2} = f_{i,j,k}
    \label{eq:poisson_fourier_space}
\end{equation}

where $\lambda_i$ and $\lambda_j$ are known eigenvalues, and ... \todo{Add symbol} means that a Fourier transform has been applied two times to a variable. One can see that the number of unknowns has decreased from 7 in \autoref{eq:discretized_poisson_equation} to just 3 in \autoref{eq:poisson_fourier_space}. 

\subsection{Discretization}
Spatial operators are discretized on an Arakawa C-grid \citep{arakawaComputationalDesignBasic1977}. The components of the velocity vector are defined at cell faces, while pressure, 

\subsection{Parallelization}
To speed up simulations or enable simulations over large domains, \acrshort{dales} is parallelized with the \acrfull{mpi}. \acrshort{mpi} provides an interface to manage the communication of data between processors. 

An \acrshort{mpi} run of \acrshort{dales} is started with a set number of \emph{tasks}, and each task is bound to a processor. In general, the number of tasks cannot exceed the number of available processors. The computational domain is then divided into as many sub-domains as there are \acrshort{mpi} tasks. 

\begin{figure}[H]
    \centering
    \includesvg[width=0.8\linewidth]{../images/drawings/dales_decompositions.svg}
    \caption{Possible domain decompositions in \acrshort{dales}. From left to right: $z$-aligned pencils, $y$-aligned slabs and $x$-aligned slabs. The colors indicate the sub-domains.}
    \label{fig:dales_domain_decomposition}
\end{figure}