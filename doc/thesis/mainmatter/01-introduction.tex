\chapter{Introduction}
\acrfull{les} is a technique used to accurately simulate complex fluid flows, of which the Earth's atmosphere is an example. Specifically, \acrshort{les} explicitly partly resolves turbulence, as opposed to traditional numerical weather and climate models. The latter use parameterizations for turbulent processes, leading to significant uncertainties \citep{schalkwijkWeatherForecastingUsing2015}. However, due to resolution requirements, \acrshort{les} is computationally too expensive to apply on large domains. 

Since it was discovered that \acrfullpl{gpu} can be used to accelerate \acrfull{ai} models, the complexity of these models has been increasing at an ever-increasing rate \citep{mittalSurveyTechniquesOptimizing2019}. This has sparked the onset of a feedback loop, where successes in \acrshort{ai} lead to improvements in \acrshort{gpu} technology, and vice versa. \acrshort{ai} models consist of a large number of parallel computations, an area where the \acrshort{gpu} excels. Fortunately, \acrshort{les} models are also comprised of such computations. Therefore, \acrshortpl{gpu} can help accelerate \acrshort{les} models and enable simulations on larger domains.

This thesis describes the process of \acrshort{gpu}-accelerating \acrshort{dales}, an \acrshort{les} code tailored for atmospheric flows, by using the \acrshort{acc} programming model. The efforts are limited to the dynamical core and moist thermodynamic routines of \acrshort{dales}. Furthermore, the model is going to be adapted such that multiple \acrshortpl{gpu} can be used simultaneously. Validation of the accelerated model will be done through an approach based on ensemble statistics. A study of the performance of the model will be done on computer systems on two different scales: a desktop system with consumer hardware, and the Snellius supercomputer, hosted at SURF\footnote{\url{https://www.surf.nl/}} on the Amsterdam Science Park. For both systems, the speedup of the accelerated model is going to be examined and on Snellius, the scaling to multiple \acrshortpl{gpu} will be tested.

The outline of this report is as follows: in \Autoref{chap:les} the theory behind the modeling of turbulent flows is discussed and a technical description of \acrshort{dales} is given. Chapter 3 will explain \acrshort{gpu} technology, and how \acrshortpl{gpu} can be used to accelerate fluid simulation codes. Then in \Autoref{chap:implementation}, the implementation of \acrshort{acc} in \acrshort{dales} will be described. Chapter 5 will present the results of the validation and performance studies. Finally, the conclusions and recommendations for future work are to be found in \Autoref{chap:conclusion_rec}.