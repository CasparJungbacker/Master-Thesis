\chapter*{Abstract}
\addcontentsline{toc}{chapter}{Abstract}

\acrfull{les} is a mathematical technique for performing simulations of turbulent flows, such as those found in the Earth's atmosphere. Compared to traditional numerical weather and climate models, \acrshort{les} is more accurate in representing turbulent processes and cloud dynamics. The computational burden of \acrshort{les}, however, has historically limited its application to relatively small domain sizes. In this work, part of the \acrshort{dales} atmospheric \acrshort{les} model was ported to \acrfullpl{gpu} using the OpenACC programming model. \acrshortpl{gpu}, originally designed for accelerating computations related to 3D computer graphics, excel at parallel computations, which are abundant in \acrshort{les} models. The performance of the \acrshort{gpu} port of \acrshort{dales} was measured on an NVIDIA RTX 3090 in a desktop workstation and an NVIDIA A100 in the Snellius supercomputer and compared to the existing \acrshort{cpu} implementation. For the \acrshort{bomex} intercomparison case, a speedup of 11.6 was achieved versus 8 \acrshort{cpu} cores on the desktop system, while on Snellius a speedup of 3.9 was observed compared to 128 \acrshort{cpu} cores. Furthermore, the existing \acrshort{mpi} parallelization of \acrshort{dales} was adapted such that multiple \acrshortpl{gpu} can be used simultaneously. The multi-\acrshort{gpu} implementation was tested on up to 64 NVIDIA A100 \acrshortpl{gpu}. This thesis represents a step towards the enhancement of the scalability of \acrshort{dales}, enabling simulations on larger domains at higher resolutions. While a substantial acceleration of \acrshort{dales} was achieved, further efforts are needed to port more components of the model to the \acrshort{gpu} to facilitate the simulation of increasingly realistic meteorological phenomena.