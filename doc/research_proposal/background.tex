\section{Background}
The atmospheric boundary layer (ABL) is a highly turbulent flow consisting of a wide range of turbulent eddies (whirling motions). Solving the Navier-Stokes equations explicitly for all scales is called direct numerical simulation (DNS). However, because the scales of these eddies ranges from millimeters to tens of kilometers, DNS is computationally very expensive, making simulations of the ABL with a reasonable domain size unfeasable \citep{moengNUMERICALMODELSLargeEddy2015}. Instead of resolving all scales, one could settle on resolving only the largest, most energetic scales explicitly and model the smaller scales. Such a simulation is called large-eddy simulation (LES). The Dutch Atmospheric Large-Eddy Simulation (DALES) model developed by \citet{heusFormulationDutchAtmospheric2010}, 


% As opposed to the Reynolds Averaged Navier Stokes (RANS) equations, which does not resolve any turbulence, and Direct Numerical Simulation (DNS), which resolves all turbulent scales, LES only resolves the largest turbulent scales which carry the most energy, while the small scales are parameterized \citep{masonLargeEddySimulationConvective1989}. 