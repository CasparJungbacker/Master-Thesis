\section{Methodology}

\subsection{OpenACC}
Computationally expensive parts of DALES will be offloaded to the GPU using OpenACC. OpenACC provides a range of \emph{compiler directives}, which are instructions to the compiler on how to compile or optimize the code. These directives are placed above loops, telling the compiler that the calculation should be done on the GPU. An example of such a loop, decorated with an OpenACC directive, can be found in \autoref{lst:accloop}. This directive-based approach offers multiple benefits over the traditional, lower-level programming models such as CUDA and OpenCL \citep{Herdman2012}:

\begin{itemize}
    \item Productivity: OpenACC requires minimal addition of lines of code to an existing codebase. This makes it possible for a programmer to accelerate large sections of a program in a relatively short amount of time. 
    \item Single source code: CUDA and OpenCL require the programmer to rewrite existing code into ``\emph{compute kernels}'' that are executed on the GPU. If it desired that a program is still able to run on CPU's, two versions of the same program have to be maintained: one version for CPU's and one version for GPU's. Needless to say, this is more error-prone than a single source code. As OpenACC directives are essentially just comments in the code, the programmer can add them to the existing CPU code. The same code can then be compiled for execution on both GPU's and CPU's, and the OpenACC directives will simply be ignored for the latter.
\end{itemize}

\begin{lstlisting}[language=Fortran, caption={Example of a Fortran loop offloaded to the GPU.}, label={lst:accloop}]
!$acc parallel loop
do k = 1, kmax
    do j = 1, jmax
        do i = 1, imax
            c(i,j,k) = a(i,j,k) + b(i,j,k)
        end do
    end do
end do
\end{lstlisting}

\subsection{Solving the Poisson equation}

In DALES, an equation for the pressure is obtained by taking the divergence of the momentum equation:

\begin{equation}
    \frac{\partial^2 \pi}{\partial x_i^2} = \frac{\partial }{\partial x_i} \left( - \frac{\partial \overline{u}_i \overline{u}_j}{\partial x_j} + \frac{g}{\theta_0}\overline{\theta}_v\delta_{i3} + \mathcal{F}_i - \frac{\partial \tau_{ij}}{\partial x_j} \right) \label{eq:pressure}
\end{equation}

\noindent \autoref{eq:pressure} is a Poisson equation for $\pi$. DALES provides two methods to solve this equation: 1) using Fast Fourier Transforms (FFT's), 2) using iterative solvers. Among these, the FFT-based solver is the most efficient. DALES has the ability to use the highly optimized Fastest Fourier Transform in the West (FFTW) library 

\subsection{Profiling}
To determine which parts of DALES would benefit the most from running on GPU's, a profile has been made 

\subsection{Verification}
